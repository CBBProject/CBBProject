\documentclass[a4paper]{report}
\usepackage{A4,Program}

\newcommand{\InstallLoc}{{\tt{INSTALL\_DIR}}}

\begin{document}

\title{\Huge \bf Images Library Tutorial}

\author{
    Odyss�e Team \\
    I.N.R.I.A\\
    BP. 93, 2004 Route des Lucioles \\
    06902 Sophia Antipolis Cedex, France \\
}

\maketitle
\thispagestyle{empty}
\clearpage

\tableofcontents
\thispagestyle{empty}
\clearpage
\addcontentsline{toc}{chapter}{\bf Introduction}
\markright{Introduction}

\chapter*{Introduction}

The Image library offers some basic functionnalities to deal with images such as creating, loading
or writting images, resizing and sampling images, indexing or iterating over pixels, filtering,
\ldots

\medskip

\noindent
The main classes provided by the library are \verb!Image1D!, \verb!Image2D! and \verb!Image3D!.
These are classes templated by a pixel type which can be any user type. In addition, there is a
common base class \verb!Image! and a generic multidimensional class \verb!BaseImage!.

\medskip

\noindent
The rest of this document shows many small examples of how the library can be used. It is not
intended to be a full reference manual, but rather a collection of small pieces of code illustrating
the capabilities of the library.

\chapter{Compiling with the library}

First, make sure you are using the correct version of the C++ compiler, usually the latest version
of g++ is better (3.2.2 at the time of this writting).

\medskip

\noindent
Hereafter, we will assume that the libraries have been installed in a given directory that will be
named \verb!INSTALL_DIR!. Typically, \verb!INSTALL_DIR!  will be something like
\verb!/proj/odyssee/home!, but this might change. To use the library, the compiler needs to know
about two kind of things: the include files and the library itself which contains the executable
code. To inform the compiler of where these files can be found, some flags (\verb!-I!, \verb!-L!,
\verb!-l!) must be passed. If you use a \verb!Makefile!, these flags can be stored in variables that
are used to create the various command lines needed for the compilation. Here are examples of the
kind of variables and flags that typically would have to be set. For more details about the various
flags and variable names, see the manuals of the compiler and \verb!make!, respectively.  Remember
to replace \verb!INSTALL_DIR! with the place where your library has been installed.

\medskip

\noindent
A typical Makefile for RawPgmIOuchar2D (section \ref{section:RawPgmIOuchar2D}) could be:
\begin{verbatim}
CXX= /usr/local/gcc-3.2/bin/g++
CPPFLAGS=-I INSTALL_DIR/include
CXXFLAGS= $(CPPFLAGS) -rdynamic -Wall -g -DBZ_DEBUG
LDFLAGS=-LINSTALL_DIR/lib/ -lImages++ -ldl
RawPgmIOuchar2D: RawPgmIOuchar2D.C
        $(CXX) $(CXXFLAGS) -o $@ $< $(LDFLAGS)
clean: 
        rm -f RawPgmIOuchar2D *.o core
\end{verbatim}

\noindent
This example typically shows the libraries needed to use the Image library. It can be seen that
\verb!libImages++!, which is the library itself depends on another library \verb!dl! (which is
a system library that allows your program to load dynamically other libraries. 
The flags \verb!DBZ_DEBUG!, \verb!-g! and \verb!Wall! are optionnal and can be used to debug/improve
your code.

\chapter{Image initialisation}

\section{Constructors}
There is the usual constructor taking the dimensions as arguments like in
\begin{verbatim}
Image2D<unsigned char> image(100, 200);
\end{verbatim}
But you can also write
\begin{verbatim}
Image2D<unsigned char> image; // size=0x0
image.resize(100, 200);
\end{verbatim}
One can also give a buffer pointer as an argument to the constructor but in this case,
the image does not own the data and is not responsible for releasing the corresponding
memory. Following is a complete example:
\program{ScaleValues.C}

\section{Image copies}

Note that a code like
\begin{verbatim}
Image2D<unsigned char> image;
Image2D<unsigned char> image2;
...
image2=image;
\end{verbatim}
will simply make the two images share the same pointer buffer
and will neither copy the sizes nor the data.
In particular, if 'image' and 'image2' have the same sizes,
they are like aliases. But, a statment like
\begin{verbatim}
delete &image;
\end{verbatim}
will {\em not} release the corresponding data and 'image2' is
still usable (reference counting is used transparently).\\

\noindent
If an explicit duplication of 'image' is needed, one should write
\begin{verbatim}
Image2D<unsigned char> image;
Image2D<unsigned char> image2;
...
image2=image.copy();
\end{verbatim}

Following is a complete example of this.
\program{Copy.C}

\chapter{Data access}

\section{How to index the pixels of an Image ?}

\subsection{Random access}
\subsubsection{Integer coordinates}
Random access to the individual pixels of an image can be done using the conventionnal \verb!operator()!.
\program{PixelAccess.C}

\subsubsection{Real coordinates}
The same way, random access with float coordinates is provided and bilinear (resp. trilinear)
interpolation is used to evaluate the value that is returned. No range checking is performed.
Be aware that because of the interpolation, the range for the $x$ direction (for example) is then
$0 \leq x \leq .dimx()-2$ and not $0 \leq x \leq .dimx()-1$, as in the integer coordinates case.
\program{BilinearInterpolation2D.C}

\subsubsection{Sub-Image access}

Any of the coordinates can be defined by a \verb!Range! to specify a range of lines along the
corresponding dimension. This allows the extraction of any king of sub-space of the original
image including full sub-images or planes or lines.
\program{SubImage.C}

\subsection{Sequential access}
Sequential access is achieved through various iterators depending on the type
of the objects on which the access is made. For uniformity, all these iterators
are named in a similar fashion: \verb!iterator<type>! where \verb!type! is the type of the
element on which we iterate. This iterator can be used to read and write the underlying data
as in the standard C++ library. To each of these iterator corresponds also an
iterator that cannot change the iterated data (thus it can only be used to read it)
and which name is \verb!const_iterator<type>!. Only the non-const versions are described hereafter.
The use of the const versions should be straightfoward given the examples below.

\medskip

Finally, note that not everything has been done (at least yet!) to ensure that these iterators
are compliant with the standard C++ various algorithms. Please report any lack of
functionnality that would be annoying.

\medskip

\emph{A current (suspected) bug in the g++ compiler prevents sometimes the use of
the} \verb!iterator<type>! \emph{notation particularly in some template functions. In such cases,
the use of the internal name which looks like} \verb!type\_\[const\_\]iterator! \emph{might
be necessary.}

\subsubsection{The pixel iterator}

The pixel iterator obviously allows to iterate over all the pixels of an image.
It's name is \verb!iterator<pixel>!.
\program{PixelIterator.C}

\subsubsection{The line iterators}

The line iterators allow iteration over the various lines of the images. These
are named \verb!iterator<line<type> >! where \verb!type! designates the type of lines
on which the iteration takes places. Valid values for \verb!type! are \verb!row!,
\verb!column! and \verb!depth! or alternatively \verb!X!, \verb!Y! and \verb!Z!.
The program below shows a small example of their usage.
\program{LineIterator.C}

\subsubsection{The slice iterators}

Slice iterators are a generalization of the line iterators that allow to iterate on
any kind of linear sub-space of the image. For example, they can be used to iterate
over the planes, lines or pixels of an image. They also give some control over the
order of iteration. For example they can be used to iterate over the pixel of a 2D
image row-wise or column-wise (see the example given below). The type returned by
dereferencing the iterator is an image of the proper dimension. Note that, internally,
line iterators are implemented using slice iterators. These iterators are named
\verb!iterator<slice<t1,t2,...> >! where there can be up to four parameters. These
parameters indicate various dimensions \verb!X!, \verb!Y! or \verb!Z! as with the
line iterators. The interpretation of these parameters is iterate over the subspace
\verb!t1=constant! first, then over the subspace \verb!t2=constant! of that space,
etc.
\program{SliceIterator.C}

\subsubsection{The domain iterators}

Domain iterators are used to walk over all the valid indexes of an image. These can be
extremely useful when one wishes to walk over various images simultaneously (with the
same indexes or with indexes transformed by some arbitrary function). The name of this
iterator is \verb!iterator<domain>!. The program below shows an extremely simple example
of their usage. Note that dereferencing this iterator provides a data structure that
can be asked for its coordinates using \verb!operator()(const Index i)! where the \verb!Index i!
designates a dimension of the space (ie this function returns the \verb!i!-th coordinate
of the current pixel referred to by the iterator. Some operations are allowed on this
indexes (comparisons, 1D-shifts, incrementation, decrementation), more should come later
(neighbors, scaling, translations, ...).
\program{DomainIterator.C}

\chapter{Handling borders}

The Image library has two means for handling access to the borders of an image (by borders
we mean the pixel that are at locations outside of the image range).

\section{Boundary images}

The first method wraps the image into a special container that systematically checks
the index given by the user:

\begin{itemize}
    \item If the index is within the range of the image, then the image value is returned.
    \item If the index is outside the range of the image, then a sensible value is provided.
\end{itemize}

Two such wrappers currently exist: with \verb!SpecialValueBoundary! the sensible value is a special
value given by the user (zero by default), whereas \verb!ConstantBoundary! just extends the image
at infinity assuming that it is constant outside of the range.

This method is particularly useful when nothing is known about the index that will be used with the
image. This is typically the case when you do matching for example, since nothing warranties that
the matching map is restricted to the range of the image.

Here is an example of how to use these wrappers.

\program{Boundary.C}

\section{Handling fixed size borders}

In many situations only the very few border pixel are accessed during a local calculation.
This is the case, for example, when evolving a spatial PDE over the image range. In this case,
one often wants to allocate a slightly bigger image containing the border and iterate only on
the sub-space consisting of the original image range. To allocate such \emph{bordered images},
the class \verb!ConstantBorderedImage! can be used, the constructor of this class takes an
image shape (its dimensions) and a border size and create an image of the desired size. However,
it keeps internally an enlarged image and provides a method \verb!UpdateBorders! that can be used
to update the border with the image content. Currently, only constant borders are supported, but
this same class might also be used to deal with fixed value borders.

\emph{Insert an example here !!!}

\chapter{Input/Output}

\section{How to read an image without specifying any type or any format ?}
\program{IOpointer.C}

\section{How to read an image of a specified type ?}
\program{IOuchar2D.C}

\section{How to read an image in a specified format ?}
\label{section:RawPgmIOuchar2D}
\program{RawPgmIOuchar2D.C}

\section{How to read a color PPM image ?}
\program{BreakPpm.C}

\section{What are the available formats ?}
\begin{itemize}
\item RawPgm
\item Inrimage
\item RawPpm
\item Pnm (which supercedes both RawPgm and RawPpm).
\end{itemize}

\section{How to write a reader/writer for the format Toto ?}

Following are the files RawPgm.H and RawPgm.C. Copy them and modify them.

\program{RawPgm.H}
\program{RawPgm.C}

As you can see a plugin must derive from \verb!ImageIO! (or one of its specialized derivatives \verb!ImageIO2D! or
\verb!ImageIO3D!). This pure abstract base class mandates the implementation of the following methods:

\begin{itemize}
    \item \verb!ImageIO* clone() const throw(std::bad_alloc)!\\
          This function just return a copy of the io structure. In the future, it might be provided by a simple base class from
          which the plugin must be derived (TODO). For now, its implementation is straitforward (see line 63 of RawPgm.H).
    \item \verb!const char* identity() const throw()!\\
          This function just returns the name of the plugin. For some reason, this name must be stored in a global static
          variable named \verb!identity! and marked as \verb!extern "C"! (see line 10 of RawPgm.H). Thus the simplest way
          to implement this methos is just to return this global variable.
    \item \verb!bool known(const char* buffer,const unsigned n) const throw()!\\
          This function is given a buffer \verb!buffer! of \verb!n! characters, this buffer contains the first \verb!n!
          characters of the file that is to be read. These characters are used to determine the file format. The function should
          return \verb!true! iff the plugin knows how to deal with this file.
    \item \verb!void identify(std::istream& is)!\\
          This function is called iff the previous function has returned \verb!true!. Its role is to read and parse the header of
          the image file and to store all the information needed to read the image (type,dimension,size). The file is accessed from
          the input stream \verb!is! which should be left pointing to the first byte of data of the image.
    \item \verb!const Id& pixel_id()  const throw()!\\
          This function is always called after the function \verb!identify!. It returns the \verb!typeid! of the pixel stored in the
          file.
    \item \verb!Dimension dimension() const throw()!\\
          This function is always called after the function \verb!identify!. It returns the dimension of the image stored in the
          file. This method is predefined in the classes \verb!ImageIO2D! (to return 2) and \verb!ImageIO3D! (to return 3) as most
          file format can handle only one type of dimension.
    \item \verb!Image* create() const throw(std::bad_alloc)!\\
          This function is always called after the function \verb!identify!. It must create an image of the appropriate type (pixel
          type, the size is adjusted afterwards) for storing the image depicted in the file.
    \item \verb!void read(std::istream& is,Image& image)        const!\\
          This function is always called after the function \verb!identify!. Its resizes the given image \verb!image! to the proper
          size and reads the data from the stream \verb!is!.
    \item \verb!bool known(const Image& image) throw()!\\
          This function must return \verb!true! iff the plugin knows how to save the image \verb!image! given as parameter.
    \item \verb!void write(std::ostream& os,const Image& image) const!\\
          This function is called iff the previous function has return true. Its role is to write the image \verb!image! to the
          stream \verb!os!.
\end{itemize}

\section{How do the input/output plugins work?}

The images readers and writers are plugins that are automatically loaded at run time by the applications using libImages++. A plugin
is a dynamic library (.so).  Plugins paths are specified in an ascii file (one per line). In the following, we refer to this file as
the "plugins list".  For example, a valid plugins list may contain the two following lines:
\begin{verbatim}
# A comment.
INSTALL_DIR/lib/robotvis/Images/plugins/libInrimage.so
INSTALL_DIR/lib/robotvis/Images/plugins/libRawPgm.so
\end{verbatim}
Lines starting with the character \# are considered as comments and are thus ignored.
A default plugins list is set at compile time, by now it is:
\begin{verbatim}
INSTALL_DIR/share/robotvis/Images/default.plugins
\end{verbatim}
One can specify an alternate plugins list by setting an environnement variable like in the following example (csh syntax):
\begin{verbatim}
setenv IMAGE_IO_PLUGINS $HOME/Images/plugins/list
\end{verbatim}
By default, the application loads the plugins silently if there is no error.  To make this process verbose, set an environnement
variable before running the application, i.e. (csh syntax):
\begin{verbatim}
setenv IMAGE_IO_PLUGINS_VERBOSE
\end{verbatim}

Normally (not tested recently), it is possible to link selected plugins statically to provide a specific set of of file formats
within the executable. Test this (TODO).

\section{Error handling}

All input/output errors are handled with exceptions. All the exceptions are described in the file Images/Exceptions.H. These
exceptions can be caught individually or by class. The base class of all image exceptions is \verb!Images::Exception!, which is used in
all the examples given in this tutorial. The exceptions classes all provide a message (method \verb!what()!) and an error code
(method \verb!code()!).

\chapter{Polymorphism}

Be aware this part has not been approved yet.

\medskip

This chapter explains how to write polymorphic code, for example how to write a program which extracts a subimage whatever the pixel
type is ?  Or how to compute the sum of two images, still doing the computations with the original pixel type, without any
conversion.  With compiled code, there is no way to escape a huge switch where the possible types are enumarated. The purpose here
is to make this as less tedious as possible ({\it i.e.} to hide the switch in the library).

\medskip

A first attempt to deal with this issue is coded in the file Polymorphic.H.  The idea is to write the code to be executed inside a
function template, the template arguments beeing the pixel type and the image dimension.  Then, the user specifies which types of
images are relevant to this code and {\em all} the corresponding functions are {\em instanciated} (of course, this is hidden).
Finally, the user provides the effective type of the image and a reference to the correct instanciated function is returned. It
remains to the user to dereference that pointer and to call the function.

\section{Example}

Here is an example of how it works.

\program{Type.C}

The declaration DeclarePolymorphicHandle declares a (very) small class able to build
instances of the Main function template: we call it a handle to that function template.

If you don't like macros, substitute this line by:

\begin{verbatim}
struct MainHandle
{
    typedef void (*prototype)(void *, int, char**);
    template <typename Pixel, unsigned Dimension>
    prototype get() { return  &Main<Pixel, Dimension>; }
};
\end{verbatim}

The function Function() returns a pointer to function (one of the many that were instanciated)
and when it is executed, an exception is thrown if something goes wrong.
The abstract type of exceptions thrown by Function() is Polymorphic::Exception (which is
a child of Images::Exception, itself a child of std::exception) and
its concrete childs are BadPixelType, BadDimension, NoImageArgument and DifferentImages.\\

For the moment, the maximal number of arguments of Types<> is 9 and that of Dimensions<> 3.

\section{Shortcuts}

It is possible to give a name to a set of types and to use this name as an argument to Types<>.
For instance, you can give the name ``NativeUnsignedInteger'' to the set of all the native unsigned
integers by writting:

\begin{verbatim}
template <>
class Type<NativeUnsignedInteger>
    : public Types<unsigned char, unsigned short, unsigned int, unsigned long> { };
\end{verbatim}

And then a switch declaration like:

\begin{verbatim}
Polymorphic::Types<Polymorphic::NativeUnsignedInteger, float, double> Switch;
...
\end{verbatim}

or, if you prefer:

\begin{verbatim}
using namespace Polymorphic;
Types<NativeUnsignedInteger, float, double> Switch;
...
\end{verbatim}
would be accepted.\\

In fact, ``NativeUnsignedInteger'' is already defined in the file Polymorphic.H, as are also defined ``NativeSignedInteger'' and ``NativeFloats''.

\end{document}
