\documentclass[10pt,a4paper]{article}

\topmargin -1.5cm
\oddsidemargin  -0.95cm
\evensidemargin -0.95cm
\textheight 25cm
\textwidth 18.0cm

\usepackage[utf8x]{inputenc}
\usepackage{amsmath,amsthm,amssymb,pifont,makeidx}
\usepackage{makeidx}
\usepackage{url}
\usepackage{graphicx}
\usepackage{color}

\title{Image Processing Projects}
\author{} 
\date{}
\begin{document}
\maketitle

\section{Initial downloads and installations}
\begin{itemize}
\item git : \url{http://git-scm.com/downloads}
\item cmake : \url{http://www.cmake.org}
\item qt 4.8.5: \url{http://qt-project.org/downloads} 
\item boost: \url{http://www.boost.org/}
\item lapack: \url{http://www.netlib.org/lapack/}
\end{itemize}
A C++ compiling environment\\
\\
Setup an account on github \url{http://www.github.com} \\
Create (if you do not have one already) an ssh key and to deposit it on this site.\\
If you succesfully installed git and setup the git account, you can
checkout the project:\\

{\tt git clone --recursive git@github.com:CBBProject/CBBProject.git}

\section{Processing of 2D gray-scale images}

\subsection{Basic Processing}

%The image class should contain:\\
%\begin{itemize}   
%\item image size (N lines, M columns)
%\item image min and max values
%\end{itemize}

\subsubsection{Load an image (already provided)} 
\begin{itemize}   
\item[Input:] image path
\item[Output:] boolean flag 
\end{itemize}
\subsubsection{Save an image (already provided)} 
\begin{itemize}   
\item[Inputs:] image path, image
\item[Output:] boolean flag 
\end{itemize}    

\subsubsection{Threshold an image}
\begin{itemize}   
\item[Inputs:] image I, scalar S
\item[Output:]thresholded image $\tt J$ such that $\tt J(i,j)=I(i,j)$ if ${\tt I(i,j)\geq S}$, otherwise ${\tt J(i,j)=0}$
\end{itemize}        


\subsubsection{Display an image}
\begin{itemize}   
\item[Inputs:] image I, \\
  image range (intensities corresponding to min (black) and max (white))
\item[Output:] display of image (thresholded to fit the range)
\end{itemize}        

\subsubsection{Multiply 2 images pixel-wise}
\begin{itemize}   
\item[Inputs:] images I and J
\item[Output:] image K such that K(i,j)=I(i,j)*J(i,j)
\end{itemize}        

\subsubsection{Add two images pixel-wise (already provided)}
\begin{itemize}   
\item[Inputs:] images I and J
\item[Output:] image K such that K(i,j)=I(i,j)+J(i,j)
\end{itemize}        

\subsubsection{Modify the pixel intensities of an image (e.g. by a look-up table)}
\begin{itemize}   
\item[Inputs:] image I\\
 name of function (look-up table) f 
\item[Output:] image J such that J(i,j) = f(I(i,j)) 
\end{itemize}       

\subsubsection{Create an image defined by a function f(x,y) }
\begin{itemize}   
\item[Inputs:] domain boundaries xmin,xmax,ymin,ymax\\
  image size N,M\\
  name of function f
\item[Output:] image J such that J(i,j) = f(i,j) 
\end{itemize}   

\subsubsection{Calculate the histogram of an image}
\begin{itemize}   
\item[Inputs:] image I\\
  number of bins B
\item[Output:] a list of B bin-centers\\
  a list of B pixel counts belonging to each bin
\end{itemize}   

\subsubsection{Subsample an image} 
\begin{itemize}     \item[Input:] image I
\item[Output:] image J such that J(i,j) = I(2*i,2*j)
\end{itemize}

\subsection{Intermediate}

\subsubsection{Convolve an image with a mask}
\url{http://en.wikipedia.org/wiki/Kernel_%28image_processing%29}

  \subsubsection{Compute 1D FFT of an image (in x or in y direction)}
  use fftw package

  \subsubsection{Compute 2D FFT of an image}
  \url{https://ia700307.us.archive.org/7/items/Lectures_on_Image_Processing/EECE253_06_FourierTransform.pdf}

  \subsubsection{Convolve an image with a Gaussian, by FFT}
  \url{https://ia700307.us.archive.org/7/items/Lectures_on_Image_Processing/EECE253_08_FrequencyFiltering.pdf}

  \subsubsection{Compute the gradient of an image (in x or in y direction)}
  \url{http://en.wikipedia.org/wiki/Image_gradient}

  \subsubsection{Compute the Laplacian of an image}
  \url{http://en.wikipedia.org/wiki/Discrete_Laplace_operator}

  \subsubsection{Filter the image with a Gaussian filter}
  \url{http://en.wikipedia.org/wiki/Gaussian_blur}

  \subsection{Advanced processing}

  \subsubsection{Compute the contours of an image}
  by using a Gaussian filter + a Laplacian filter

  \subsubsection{Subsample an image while avoiding aliasing}
  by preprocessing with a Gaussian filter, see   \url{https://ia600307.us.archive.org/7/items/Lectures_on_Image_Processing/EECE253_11_SamplingAliasing.pdf}

  \subsubsection{Compute the median filter of an image}
  \url{https://archive.org/download/Lectures_on_Image_Processing/EECE253_16_MedianFilters.pdf}

\end{document}
